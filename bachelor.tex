\documentclass{scrbook}
\usepackage{fontspec}

\author{Marek Kubica}
\title{Bachelor Thesis}
\date{\today}

\setmainfont{Linux Libertine O}
\setsansfont{Linux Biolinum O}
\setmonofont{Droid Sans Mono Dotted}

\begin{document}
\maketitle

I assure the single handed composition of this bachelor's thesis only supported
by declared resources.

%\begin{abstract}
%TODO
%\end{abstract}

\tableofcontents

\chapter{Introduction}
\label{sec:intro}

Processing data is the use-case computers were invented for: doing calculations
that are too complicated for manual computation or too time-consuming.
Therefore, in the history of computing the representation of data was always
very important. Usual hardware architectures used only to operate on scalar
values, because the registers of CPUs could only store scalar values.

Computers are very good at repeating tasks, so it is only natural to extend the
processing of one record to processing multiple records on which the
computation is repeated. Early computers worked on data in a batch structure,
by reading a record from punch cards, computing the result and writing the
result to punch cards. Later systems used their built-in memory to hold the
data and multiple ways to organize this data were possible.

One way to organize was using fixed-size arrays of values on which one
particular operation was executed. Languages organizing data this way are APL
and its sucessors J and K for example. Another example is the x86 assembly
language with its special-purpose MMX and SSE extensions which compute on sets
of registers holding floating point values.

An alternative is the list abstraction. Seemingly similar to arrays, the
high-level difference with lists is that lists do not usually have a fixed
length. While arrays are usually implemented as sequence of data in memory,
list implementations can be very flexible with different advantages and
disadvantages. A very important language family that utilizes lists heavily is
Lisp where even the source code is written in a list structure. Usually, all
modern languages provide a convenient way to handle lists, many also provide
syntactic extensions.

In functional programming in particular, lists have been very important. TODO

An evolution of lists is the sequence abstraction. Lists are limited by the
available memory to hold all the records, whereas a sequence does not need to.
To make this possible, sequences sacrifice some features of lists like random
access on each record and instead provide a way to get the first element.

TODO

\chapter{Overview of existing solutions}
\label{sec:solutions}

\section{C++}
\section{Java}
\section{Python}
\section{Lisp}
\section{Haskell}

\chapter{Inplementation}
\label{sec:implementation}

\chapter{Discussion}
\label{sec:discussion}

\bibliographystyle{plain}
\bibliography{bachelor}
\end{document}
